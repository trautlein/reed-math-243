%Header
    \documentclass[11pt]{article}
    \usepackage{indentfirst}
    \usepackage{fullpage}
    \usepackage{microtype}
    \usepackage{enumerate}
    \usepackage{amsmath}
    \usepackage{courier}

    \begin{document}    
    
\noindent{\large
    Hans Trautlein          \hfill Problem Set 2: Chapter 3\ \ \\
    Due: February 5, 2016   \hfill Andrew Bray}
\bigskip

\section*{Book Problems}


\begin{enumerate}
	\item \textit{Describe the null hypotheses to which the p-values given in Table 3.4 correspond. Explain what conclusions you can draw based on these p-values. Your explanation should be phrased in terms of \texttt{sales}, TV, radio, and newspaper, rather than in terms of the coefficients of the linear model.}
	
  	\setcounter{enumi}{4}

	\item \textit{Consider the fitted values that result from performing linear regres- sion without an intercept. In this setting, the $i$th fitted value takes the form}
	
	$$ \hat{y}_i = x_i\hat{\beta},$$
	
	\textit{where}
	
	$$ \hat{\beta} = \left(\sum_{i=1}^nx_iy_i\right) / \left(\sum_{i'=1}^nx_{i'}^2\right).$$
	
	%\sum_{i=1}^nx_iy_i \right  \left \sum_{i=1}^nx_{i'}^2
	
	\textit{Show that we can write}
	
	$$ \hat{y}_i = \sum_{i'=1}^na_{i'}y_{i'}.$$
	
	\textit{What is $a_{i'}$?}
	
	\textit{Note: We interpret this result by saying that the fitted values from linear regression are linear combinations of the response values.}
	
	\item \textit{Using (3.4), argue that in the case of simple linear regression, the
least squares line always passes through the point (x?, y?).
}
	
\end{enumerate}


\section*{Challenge Problem}
Use the identities for expected value and variance to derive the bias-variance decomposition of 

$$ E\left[ \left(y - \hat{f}(x) \right)^2 \right]$$


\end{document}

